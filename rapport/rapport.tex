\documentclass[10pt, a4paper]{article}
\usepackage[utf8]{inputenc}
\usepackage{listings}
\usepackage{hyperref}

\usepackage{fancyhdr}
\pagestyle{fancy}
\lhead{PERALE Thomas et RUSU George}
\rhead{INFO-F-203}

\usepackage{graphicx}
\usepackage{tikz}
\usepackage{color} or \usepackage{xcolor}

\lstset{ %
backgroundcolor=\color{white},   % choose the background color; you must add \usepackage{color} or \usepackage{xcolor}
basicstyle=\footnotesize,        % the size of the fonts that are used for the code
breakatwhitespace=false,         % sets if automatic breaks should only happen at whitespace
breaklines=true,                 % sets automatic line breaking
captionpos=b,                    % sets the caption-position to bottom
commentstyle=\color{mygreen},    % comment style
deletekeywords={\ldots},            % if you want to delete keywords from the given language
escapeinside={\%*}{*)},          % if you want to add LaTeX within your code
extendedchars=true,              % lets you use non-ASCII characters; for 8-bits encodings only, does not work with UTF-8
frame=single,                      % adds a frame around the code
keepspaces=true,                 % keeps spaces in text, useful for keeping indentation of code (possibly needs columns=flexible)
keywordstyle=\color{blue},       % keyword style
language=Octave,                 % the language of the code
otherkeywords={*,\ldots},           % if you want to add more keywords to the set
numbers=left,                    % where to put the line-numbers; possible values are (none, left, right)
numbersep=5pt,                   % how far the line-numbers are from the code
numberstyle=\tiny\color{mygray}, % the style that is used for the line-numbers
rulecolor=\color{black},         % if not set, the frame-color may be changed on line-breaks within not-black text (e.g.  comments (green here))
showspaces=false,                % show spaces everywhere adding particular underscores; it overrides 'showstringspaces'
showstringspaces=false,          % underline spaces within strings only
showtabs=false,                  % show tabs within strings adding particular underscores
stepnumber=2,                    % the step between two line-numbers.  If it's 1, each line will be numbered
stringstyle=\color{mymauve}, % string literal style
tabsize=2, % sets default tabsize to 2 spaces
title=\lst
}

\begin{document}
\section{Description du problème}
    Dans un parking similaire à un est tableau de plusieurs cases,
    où chaques cases peut contenir soit une voiture qui peut avancer
    ou reculer selon son orientation (\emph{verticale} ou \emph{horizontale})
    soit un espace vide, déterminer pour une des voitures de ce parking
    (la voiture \emph{goal}) comment atteindre la sortie de ce parking en
    faisant bouger les voitures le moins possible.

\section{Cas de base.}
    Est fourni comme input pour le problême un fichier: \newline
    \lstinputlisting{../test/test1.txt}
    Il nous donne des informations sur:
    \begin{itemize}
        \item La taille du parking.
        \item La position de la sortie.
        \item L'emplacement des voitures dans ce parking.
    \end{itemize}
    Qui vont être parsé pour pouvoir créer un parking de base, qui est notre
    cas de base à partir duquel il va falloir trouver le plus court chemin.

\section{Description de la résolution du problême.}
    Le problême va se faire en plusieurs étapes:
    \begin{enumerate}
        \item Générer tout les configurations\footnote{Une configuration de
            parking est un parking avec les voitures à une certaine place.
            Quand on dit toutes les configurations possible on entend
            qu'à chaque fois qu'une voiture va bouger dans le parking on a une
            configuration différente.} de parking possible.
        \item Lors de cette génération, créer un graphe dans lequel chaques
            configurations nouvellement crée est lier à la configuration
            d'où elle vient.
        \item Une fois ces configurations générées, appliquer un algorithme de
            recherche du plus court chemin sur le graphe des configurations.
    \end{enumerate}

    Exemple\footnote{Les exemples sont créés à partir de
    http://analogbit.com/software/puzzletools/ .} de graphe associer aux
    configurations lors d'une génération non exhaustive (on abouti à deux
    situation gagnant2).\newline

    % \includegraphics[scale=0.2]{gen0.png} \newline
    \newpage
    \begin{tikzpicture}
        [scale=.8,auto=left,every node/.style={circle,fill=blue!20}]
        \node (n0) at (1, 10) {\includegraphics[scale=0.15]{gen0.png}};
        \node (n1) at (5, 10) {\includegraphics[scale=0.15]{gen1.png}};
        \node (n2) at (9, 10) {\includegraphics[scale=0.15]{gen2.png}};
        \node (n3) at (13, 10) {\includegraphics[scale=0.15]{gen3.png}};
        % Premier win

        %Autre situation à partir de n0
        \node (n4) at (1, 7) {\includegraphics[scale=0.15]{gen4.png}};
        \node (n5) at (4, 7) {\includegraphics[scale=0.15]{gen5.png}};
        \node (n6) at (7, 7) {\includegraphics[scale=0.15]{gen6.png}};
        \node (n7) at (10, 7) {\includegraphics[scale=0.15]{gen7.png}};
        \node (n8) at (13, 7) {\includegraphics[scale=0.15]{gen8.png}};
        \node (n9) at (13, 4) {\includegraphics[scale=0.15]{gen9.png}};

        \node (n10) at (13, 4) {\includegraphics[scale=0.15]{gen10.png}};

        \foreach \from/\to in {n0/n1, n1/n2, n2/n3, n0/n4, n4/n5, n5/n6, n6/n7,
        n7/n8, n8/n9}
        \draw (\from) -- (\to);
    \end{tikzpicture}

\subsection{Génération des configurations.}
\subsection{Recherche du plus court chemin.}

\begin{thebibliography}
\bibitem{rushhourdijkstra}
    Mark Stamp, Brad Engel McIntosh Ewekk, Victor Morrow,
    \emph{Rush Hour and Dijkstra's algorithm}
    http://www.cs.sjsu.edu/~stamp/cv/papers/rh.pdf

\end{thebibliography}

\end{document}
